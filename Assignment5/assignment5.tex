\documentclass[12pt,letter]{article}
\usepackage{geometry}\geometry{top=0.75in}
\usepackage{amsmath}
\usepackage{amssymb}
\usepackage{mathtools}
\usepackage{xcolor}	% Color words
\usepackage{cancel}	% Crossing parts of equations out
\usepackage{tikz}    	% Drawing 
\usepackage{pgfplots}   % Other plotting
\usepgfplotslibrary{colormaps,fillbetween}
\usepackage{placeins}   % Float barrier

% Don't indent
\setlength{\parindent}{0pt}
% Function to replace \section with a problem name specifically formatted
\newcommand{\problem}[1]{\vspace{3mm}\Large\textbf{{Problem {#1}\vspace{3mm}}}\normalsize\\}
% Formatting function, like \problem
\newcommand{\ppart}[1]{\vspace{2mm}\large\textbf{\\Part {#1})\vspace{2mm}}\normalsize\\}
% Formatting 
\newcommand{\condition}[1]{\vspace{1mm}\textbf{{#1}:}\normalsize\\}

\begin{document}
\title{CIS 621 Assignment 5}
\author{Steven Walton}
\maketitle
\problem{1}
For the ``ski rental" problem, suppose renting the ski costs \$1 for Day 1 and 
buying the ski costs \$p, where $p\in\mathbb{Z}^+$ (the positive integers)
and $p\gg1$. It is already known that, if the rental price stays \$1 for every
day, the best competitive ratio for any deterministic online algorithm is
$c_\mathsf{static} = 2-\frac1p$. Suppose the rental price can vary arbitrarily 
in $\mathbb{Z}^+$ since Day 2, prove that the best competitive ratio for any 
deterministic online algorithm is $c_\mathsf{dynamic} = p$.
\\

\textbf{Hint}: Consider an online algorithm $A_d$ that keeps renting the ski 
until buying it on the d$^{th}$ day, where $d\in\mathbb{Z}^+$. Think about
how the ``adversary" (or ``environment") can respond to $A_d$. Maybe study the
cases of $d=1$ and $d\geq 2$, respectively, and then summarize.
\\

\textbf{Solution}:\\
Our goal is to minimize our cost, or maximize the distance between the rental 
price and the price of the skis. That is
\[
    \max\|f(rent)-p\|
\]
Conversely the renter wants to maximize the distance (maximize their profits)
\[
    \min\|f(rent)-p\|
\]
$\therefore$ our best option is to actually buy the skis on the first day, as
this maximizes our cost distance. But the optimal solution is renting for
\$1 at day one. So we have
\begin{align*}
    c_{\mathsf{dynamic}} &=\frac{\max\|f(rent)-p\|}{1}\\
    &= \frac{p}{1}\\
    &= p
\end{align*}




\end{document}
