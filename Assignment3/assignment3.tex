\documentclass[12pt,letter]{article}
\usepackage{geometry}\geometry{top=0.75in}
\usepackage{amsmath}
\usepackage{amssymb}
\usepackage{mathtools}
\usepackage{xcolor}	% Color words
\usepackage{cancel}	% Crossing parts of equations out
\usepackage{tikz}    	% Drawing 

% Don't indent
\setlength{\parindent}{0pt}
% Function to replace \section with a problem name specifically formatted
\newcommand{\problem}[1]{\vspace{3mm}\Large\textbf{{Problem {#1}\vspace{3mm}}}\normalsize\\}
% Formatting function, like \problem
\newcommand{\ppart}[1]{\vspace{2mm}\large\textbf{\\Part {#1})\vspace{2mm}}\normalsize\\}
% Formatting 
\newcommand{\condition}[1]{\vspace{1mm}\textbf{{#1}:}\normalsize\\}

\begin{document}
\title{CIS 621 Assignment 2}
\author{Steven Walton}
\maketitle
\problem{1}
Consider $f:\mathbb{R}^n \rightarrow \mathbb{R}, f(x) = \|x\|_2$. Show that 
\[ 
g = 
\begin{cases}
    \frac{x}{\|x\|_2} & x \neq 0\\
    \{y | \|y\|_2 \leq 1\} & x = 0
\end{cases}
\]

As usual, we will just take the derivative of $f$. We know this is what to do
because of the definition of subgradients. Again, we use Einsteinian notation.
\begin{align*}
    f(x) &= \|x\|_2 \\
         &= \sqrt{x_i^2} \\
    \frac{\partial f}{\partial x_i} &= (\cancel{2}x_i)\left(\cancel{\frac12}\frac{1}{\sqrt{x_i^2}}\right) \\
         &= \frac{x_i}{\sqrt{x_i^2}} \\
    \nabla f&= \frac{x}{\|x\|_2}
\end{align*}
Therefore we can see that any gradient of the function is a subgradient. We 
should also recognize that this is the same as $norm\{x\}$. 

Dealing with the case where $x=0$ we can see that the derivative is 
$\{g | \|g\|_2 \leq 1\}$. $\therefore$ we have our solution.

\problem{2}
Consider $f(x) = \max\{f_1(x),f_2(x)\}$ where $f_1,f_2:\mathbb{R}^n\rightarrow
\mathbb{R}$ are convex and differentiable. Show that for $f_1(x) > f_2(x)$ the
subgradient is $\nabla f_1(x)$, for $f_1(x) < f_2(x)$ the subgradient is
$\nabla f_2(x)$, and for $f_1(x) = f_2(x)$ the subgradient is any point on the
line segment between $\nabla f_1(x), \nabla f_2(x)$.

This problem is fairly straight forward. Remembering back to the definition of
subgradients we may recall, again, that when $f$ is differentiable then $\nabla f$
is the only possible choice for $g$. Clearly from there we can see
\begin{align*}
    f_1(x) > f_2(x) & \\
        & \max\{f_1(x),f_2(x)\} = f_1(x) \\
        & \Rightarrow g(x) = \nabla f_1(x) \\
    f_1(x) < f_2(x) & \\
        & \max\{f_1(x),f_2(x)\} = f_2(x) \\
        & \Rightarrow g(x) = \nabla f_2(x)
\end{align*}
The trick is that when $f_1 = f_2$ then both are acceptable solutions to the 
$\max$ function. Knowing this we can see that $g=\nabla f_1$ and $g=\nabla f_2$ 
are acceptable solutions. $\therefore$ we can conclude that any solution 
between these two solutions is also true. Or more explicitly
\[
    g(x) = \alpha \nabla f_1(x) + \beta \nabla f_2(x)
\]
where $\alpha + \beta = 1$

\problem{3}
Use the definition of subdifferentials to show that the following two functions 
are not subdifferentiable at $x=0$. (1) $f(x) = -x^\frac12$ (2) $f(x)$ s.t.
$f(0) = 1$ and $f(x) = 0, x>0$

\ppart{1}
\begin{figure}[h]
\centering
\begin{tikzpicture}
      \draw[<->] (-1,0) -- (5,0) node[right] {$x$};
      \draw[<->] (0,-2.5) -- (0,1) node[above] {$y$};
      \draw (-1, 0) circle (0pt) node[above] {undefined};
      \draw[->,domain=0:5,ultra thick,smooth,variable=\x] plot ({\x},{-sqrt(\x)});
\end{tikzpicture}
\caption{$-x^\frac12$}
\end{figure}

\begin{align*}
    \nabla f &= -\frac12 x^\frac{-1}{2} \\
             &= -\frac12\frac{1}{\sqrt{x}}
\end{align*}
Trivially we can see that when $x=0$ this function is not defined. 

To show that it has no subdifferential we need to note that 
$\lim\limits_{x\rightarrow 0^-}f(x) \neq \lim\limits_{x\rightarrow 0^+}f(x)$ because 
$\lim\limits_{x\rightarrow 0^-}f(x)$ is undefined and $\notin \mathbb{R}$.
$\therefore$ there is no subdifferential. (While a subdifferential does not 
need to be differentiable at a point, its limit must exist and be well defined)

\ppart{2}
\begin{figure}[h]
\centering
\begin{tikzpicture}
      \draw[<->] (-1,0) -- (3,0) node[right] {$x$};
      \draw[<->] (0,-1) -- (0,1.5) node[above] {$y$};
      \fill (0,1) circle (2pt);
      \draw (0,0) circle (2pt);
      \draw (-1, 0) circle (0pt) node[above] {undefined};
      \draw[->,domain=0:2.5,ultra thick,smooth,variable=\x] plot ({\x},{0});
\end{tikzpicture}
\caption{$f(0) = 1, f(x) = 0 \forall x>0$}
\end{figure}
This functions is not differentiable, because
this function has is discontinuous at $x=0$. Similar to part 1 we may notice that the
left limit is undefined, and in this case doesn't exist in any space. $\therefore$
there is no subdifferential.

\problem{4}
Consider the subgradient method $x^+=x-\alpha g$ where $g\in\nabla f(x)$. Show
that if $\alpha < \frac{2(f(x) - f(x^*))}{\|g\|_2^2}$, then $\|x^+-x^*\|_2 
< \|x-x^*\|_2$. (Every iteration moves closer to the optimal $x^*$)

\begin{align*}
\|x^+ - x^*\|_2^2 &= \|x-\alpha g - x^*\|_2^2 \\
		  &= \|x-x^*\|_2^2 - 2\alpha g(x-x^*) + \alpha^2 \|g\|_2^2\\
		  &< \|x-x^*\|_2^2 - 2\alpha(f(x)-f(x^*)) + 2\left(\frac{(2(f(x)-f(x^*)))^2}{(\|g\|_2^2)^{\cancel{2}}}\right)\cancel{\|g\|_2^2}\\
		  &< \|x-x^*\|_2^2 - 2\alpha(f(x)-f(x^*)) + 4(f(x)-f(x^*))\alpha\\
		  &< \|x-x^*\|_2^2 - 6\alpha(f(x)-f(x^*))
\end{align*}
From here we can see that because of the extra term and the upper bound provided
to $\alpha$ we can conclude that $\|x^+-x^*\| < \|x-x^*\|$

\problem{5}
Use ``lagrange relaxation" and ``dual decomposition" to design a distributed 
algorithm to find the optimal value of the objective function of the following 
problem and describe your distributed algorithm elaborately. 
$a_{s,u}, b_{s,u}, c_u, d_s, \forall s\in \mathcal{S}, \forall u \in \mathcal{U}$
are all non-negative constants
\begin{align*}
\min\hspace{3mm}&\hspace{-3mm}\sum\limits_{s\in\mathcal{S},u\in\mathcal{U}}a_{s,u}x_{s,u} + 
	\sum\limits_{s\in\mathcal{S},u\in\mathcal{U}}b_{s,u}((x_{s,u}+1)\log(x_{s,u}+1)-x_{s,u})\\
s.t.\hspace{3mm} &\sum\limits_{s\in\mathcal{S}}x_{s,u}\geq c_u, \forall u\in\mathcal{U}\\ 
     &\sum\limits_{u\in\mathcal{U}}x_{s,u}\geq d_s, \forall s\in\mathcal{S}\\
     &x_{s,u}\geq 0, \forall s\in\mathcal{S}, \forall u\in\mathcal{U}
\end{align*}

We will rewrite this as
\begin{align*}
\min\hspace{3mm}&\hspace{-3mm}\sum\limits_{s\in\mathcal{S},u\in\mathcal{U}}a_{s,u}x_{s,u} + 
	\sum\limits_{s\in\mathcal{S},u\in\mathcal{U}}b_{s,u}((x_{s,u}+1)\log(x_{s,u}+1)-x_{s,u})\\
s.t.\hspace{3mm} &c_u - \sum\limits_{s\in\mathcal{S}}x_{s,u}\leq 0, \forall u\in\mathcal{U}\\ 
     &d_s - \sum\limits_{u\in\mathcal{U}}x_{s,u}\leq 0, \forall s\in\mathcal{S}\\
     &-x_{s,u}\leq 0, \forall s\in\mathcal{S}\\
     &-x_{s,u}\leq 0, \forall u\in\mathcal{U}
\end{align*}
From here we can see that we clearly have two separable problems, one over the
domain for $\mathcal{S}$ and another over $\mathcal{U}$.
\begin{align*}
\min\hspace{3mm}&\hspace{-3mm}\sum\limits_{s\in\mathcal{S}}a_{s,u}x_{s,u} + 
	\sum\limits_{s\in\mathcal{S}}b_{s,u}((x_{s,u}+1)\log(x_{s,u}+1)-x_{s,u}) \\
	&+ \lambda_2 \left(d_s - \sum\limits_{u\in\mathcal{U}}x_{s,u} -x_{s,u}\right)\\
 s.t.\hspace{3mm}    &c_u - \sum\limits_{u\in\mathcal{U}}x_{s,u}\leq 0, \forall u\in\mathcal{U}\\
     &-x_{s,u}\leq 0, \forall u\in\mathcal{U}\\
\end{align*}
and
\begin{align*}
\min\hspace{3mm}&\hspace{-3mm}\sum\limits_{u\in\mathcal{U}}a_{s,u}x_{s,u} + 
	\sum\limits_{u\in\mathcal{U}}b_{s,u}((x_{s,u}+1)\log(x_{s,u}+1)-x_{s,u}) \\
	&+ \lambda_1 \left(c_u - \sum\limits_{s\in\mathcal{S}}x_{s,u} -x_{s,u}\right)\\
 s.t.\hspace{3mm}    &d_s - \sum\limits_{u\in\mathcal{U}}x_{s,u}\leq 0, \forall s\in\mathcal{S}\\
     &-x_{s,u}\leq 0, \forall s\in\mathcal{S}\\
\end{align*}

Where $\lambda_1, \lambda_2 \geq 0$\\
$\because$ Lagrangian Relaxation, we are able to solve these equations faster 
and they are separable over the two different domains and $\therefore$ 
parallelizeable. 

To find an optimal value we will need to use an iterative method, such as
gradient decent, steepest decent, or Newton's method. 
To create an effective algorithm we would need to pass back data per iteration.
Each domain would be calculated separately, passed back, and the be added together.
Then the next step can take place in the same manner.

\end{document}
