\documentclass[12pt,letter]{article}
\usepackage{geometry}\geometry{top=0.75in}
\usepackage{amsmath}
\usepackage{amssymb}
\usepackage{mathtools}
\usepackage{xcolor}	% Color words
\usepackage{cancel}	% Crossing parts of equations out
\usepackage{tikz}    	% Drawing 

% Don't indent
\setlength{\parindent}{0pt}
% Function to replace \section with a problem name specifically formatted
\newcommand{\problem}[1]{\vspace{3mm}\Large\textbf{{Problem {#1}\vspace{3mm}}}\normalsize\\}
% Formatting function, like \problem
\newcommand{\ppart}[1]{\vspace{2mm}\large\textbf{\\Part {#1})\vspace{2mm}}\normalsize\\}
% Formatting 
\newcommand{\condition}[1]{\vspace{1mm}\textbf{{#1}:}\normalsize\\}

\begin{document}
\title{CIS 621 Assignment 4}
\author{Steven Walton}
\maketitle
\problem{1}
For the graph $G=(\mathcal{U},\mathcal{E})$, where $\mathcal{U}$ is the set of 
vertices and $\mathcal{E}$ is the set of edges, we define the following
nonlinear integer program, where $w_{i,j}\geq0, \forall(i,j)\in\mathcal{E}$ and 
$k$ is a nonnegative integer:
\begin{align*}
    \sup&\sum\limits_{(i,j)\in\mathcal{E}} w_{i,j}(x_i + x_j - 2x_ix_j)\\
    %\hspace{-3.5cm}s.t. & \sum\limits_{i\in\mathcal{U}}x_i = k\\
    s.t. & \sum\limits_{i\in\mathcal{U}}x_i = k\\
       & x_i\in\{0,1\}, \forall i \in\mathcal{U}
\end{align*}
Show that the following linear program is a relaxation of the above problem:
\begin{align*}
    \sup&\sum\limits_{(i,j)\in\mathcal{E}} w_{i,j}z_{i,j}\\
     s.t.\hspace{1cm} & z_{i,j}\leq x_i + x_j, \forall(i,j)\in\mathcal{E}\\
     & z_{i,j}\leq 2 - x_i - x_j, \forall(i,j) \in \mathcal{E}\\
     & \sum\limits_{i\in\mathcal{U}}x_i = k\\
     & 0 \leq x_i \leq 1, \forall i \in\mathcal{U}\\
     & 0 \leq z_{i,j} \leq 1, \forall(i,jP\in\mathcal{E}
\end{align*}
Also, let $F(x)=\sum\limits_{(i,j)\in\mathcal{E}}w_{i,j}(x_i + x_j - 2x_ix_j)$ be
the objective function of the nonlinear integer program. Show that for any 
$(x,z)$ that is feasible to the linear program, 
$F(x)\geq\frac12\sum\limits_{(i,j)\in\mathcal{E}}w_{i,j}z_{i,j}$
\ppart{1}
If we remember the definition of linear relaxation we see that it is
$x_i\in\{0,1\} \mapsto 0\leq x_i \leq 1$



\end{document}
